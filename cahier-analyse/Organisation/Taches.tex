\subsection{Répartition des tâches}

Afin d'assurer le bon déroulement du développement du jeu, les tâches ont été distribuées parmi les membres de l'équipe en fonction de leurs compétences et préférences. Cette répartition a été supervisée par le chef de projet, qui a attribué les responsabilités de manière stratégique. Grâce à l'outil GitHub Project, nous avons pu suivre l'avancement de chaque membre, particulièrement lors du travail à distance. Les tâches ont été organisées en modules, chacun correspondant à une phase du projet. Chaque membre a pris en charge un ou plusieurs modules, permettant ainsi de travailler en parallèle sur différents aspects du jeu et d'accélérer son développement. La répartition des tâches a été équilibrée et équitable, assurant une contribution significative de chaque membre à la réalisation du projet. \\

\textbf{PUREN Mewen} : Chef de projet, développement du système de sauvegarde, gestion des utilisateurs, définition des conventions de codage et configuration de GitHub, répartition des tâches au sein de l’équipe, paramétrage de la compilation Maven. \\

\textbf{PESANTEZ Maelig} : Documentaliste, mise en place de la JavaDoc web, gestion des activités sur GitHub, création des assets graphiques, développement de l’affichage JavaFX (visualisation du jeu, tutoriel), élaboration des livrables (cahier des charges, cahier d’analyse). \\

\textbf{COGNARD Luka} : Développement de la logique du jeu, assistance pour les aides contextuelles, supervision des actions sur GitHub. \\

\textbf{MABIRE Aymeric} : Développement de la logique du jeu, développement de l’affichage JavaFX (affichage du jeu, gestion des assets, liaison des menus),développement des aides, rédaction du manuel utilisateur, création du diagramme de classes. \\

\textbf{TOUISSI Nassim} : Développement des menus, création du diagramme de GANTT. \\

\textbf{LELANDAIS Clément} : Développement des aides, relecture du cahier des charges. \\

\textbf{GAUMONT Mael} : Développement des aides.
