\subsection{Javadoc}

La documentation \emph{Javadoc} de notre projet est disponible en ligne sur \href{https://gr4-m3acnl.github.io/hashi/}{GitHub}. Afin de garantir son exhaustivité, nous avons mis en place des règles de compilation via Maven, rendant obligatoire l'ajout de commentaires au format \emph{Javadoc} pour chaque classe, méthode et variable, qu'elle soit de classe ou d'instance.  

Cette exigence, appliquée dès le début du projet, assure une documentation continue du code et présente plusieurs avantages :

\begin{itemize}
    \item \textbf{Lisibilité et compréhension} : Les commentaires Javadoc expliquent le fonctionnement et l'utilisation du code, facilitant ainsi sa compréhension.
    \item \textbf{Amélioration du travail en équipe} : Une documentation claire favorise la communication entre les développeurs et simplifie la collaboration.
    \item \textbf{Maintenance et évolutivité} : Un code bien documenté est plus simple à maintenir et à faire évoluer au fil du temps.
    \item \textbf{Réutilisation du code} : Une documentation explicite encourage la réutilisation des éléments du projet en fournissant des indications précises sur leur usage.
\end{itemize}

L'intégration de la Javadoc dans notre flux de développement contribue ainsi à renforcer la qualité, la transparence et la fiabilité de notre projet.
