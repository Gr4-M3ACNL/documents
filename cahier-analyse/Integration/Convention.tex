\subsection{Conventions}  

\hypertarget{checkstyledescription}{  
    Afin de garantir une intégration fluide et une collaboration efficace entre les membres de l’équipe, nous avons défini des conventions de codage et de documentation à respecter tout au long du développement. Ces règles sont spécifiées dans un fichier \texttt{checkstyle.xml}, qui permet d’en vérifier l’application lors de la compilation du code. Cette validation est effectuée automatiquement par Maven à l’aide du plugin Checkstyle, et toute non-conformité entraîne un blocage de la compilation. Bien que certains membres de l’équipe aient rencontré des difficultés à s’adapter à ces exigences, l’usage de Checkstyle leur a permis de mieux appréhender l’importance de la qualité du code et de la documentation, tout en améliorant progressivement leurs pratiques. \\  
}  

Concernant l’utilisation de GitHub, nous avons mis en place des règles précises pour la gestion des branches, des commits et des pull requests. Le dépôt repose sur deux branches principales : \texttt{master} et \texttt{develop}. La branche \texttt{master} contient la version stable de l’application et correspond à la version de production, tandis que la branche \texttt{develop} regroupe les développements en cours, qui doivent rester aussi stables que possible. Toute nouvelle branche doit être créée à partir de \texttt{develop}. Chaque membre de l’équipe est tenu de créer une branche spécifique pour chaque nouvelle fonctionnalité et d’effectuer un commit à chaque modification significative. \\  

Les messages de commit doivent être clairs et explicites, décrivant précisément les modifications apportées. Toute nouvelle branche doit faire l’objet d’une pull request, qui doit être revue et approuvée par un autre membre de l’équipe avant d’être fusionnée dans la branche principale. Ces bonnes pratiques ont permis de garantir une intégration continue et une collaboration efficace tout au long du projet. \\  
