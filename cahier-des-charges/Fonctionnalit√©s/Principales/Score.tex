\fonction{Score}
{
    Le score d'une partie correspond à son chronomètre, c'est à dire le temps que le joueur a mis pour résoudre la grille avec des pénalités liées aux aides utilisées.
}
{
    Le score vise à instaurer une compétition autant à soi-même qu'entre les joueurs, tout en offrant une mesure de leurs performances. Le score sera attribué en fonction des décisions prises pendant le jeu, notamment l'utilisation d'aides.
    \\
    \\
    L'utilisation d'aides pendant le jeu entraînera une augmentation du score. Plus le joueur aura recours à des indices ou à d'autres formes d'aide, plus son chronomètre se verra pénalisé par des malus. Cela encourage le joueur à résoudre le puzzle de façon autonome et optimale pour maximiser son score.
    \\
    \\
    Points réduits en fonction des aides :
    \noexpand\begin{itemize}[label=$\Rightarrow$]
        \noexpand\item Vérification des erreurs :
            \noexpand\begin{itemize}
                \noexpand\item Demande d'aide de niveau un (technique) : ajoute une seconde au chronomètre
                \noexpand\item Demande d'aide de niveau deux (zone) : ajoute deux secondes au chronomètre
                \noexpand\item Demande d'aide de niveau trois (nœuds concernés) : ajoute trois secondes au chronomètre
                \noexpand\item Retour au checkpoint (mode hypothèse) : le chronomètre n'est pas affecté
            \noexpand\end{itemize}
    \noexpand\end{itemize}
}