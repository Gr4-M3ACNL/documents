\fonction{Sauvegarde}
{
    L'application sauvegarde de nombreux éléments afin que le joueur puisse quitter et revenir sans perdre sa progression dans les différents niveaux.
    \begin{itemize}
        \item \subfonction{Grille}
            {La progression sur la complétion des grilles est sauvegardée}
            {
                Lors d'une partie, l'application sauvegarde tous ce qui permettra au joueur de reprendre sa partie là où il en était. La sauvegarde étant automatique, l'utilisateur peut quitter une partie à tout moment sans se préoccuper de conserver sa progression. Des informations complémentaires à la grille sont sauvegardés telles que :
                \noexpand\begin{itemize}
                    \noexpand\item L'historique des coups joués pour permettre les retours en arrière
                    \noexpand\item Le score actuel de la partie
                    \noexpand\item L'aide des combinaisons possibles déjà activée ou non
                \noexpand\end{itemize}
                Il y a une sauvegarde pour chaque niveau commencé.
                Le joueur reprendra automatiquement où il s'est arrêté pour chaque niveau ayant une sauvegarde.
            }
        \item \subfonction{Options}
            {Les options choisies par l'utilisateur sont sauvegardées.}
            {
                Toutes les modifications faites dans les options du jeu sont sauvegardées sur le profil, évitant ainsi de modifier à nouveau lors d'une connexion ultérieure.
            }
        \item \subfonction{Autres}
            {D'autres données sont également sauvegardées}
            {
                D'autres données sont également sauvegardées pour chaque utilisateur:
                \noexpand\begin{itemize}
                    \noexpand\item Le temps total de jeu.
                    \noexpand\item Les fenêtres déjà visitées comme le tutoriel pour ne pas répéter les introductions à la découverte du jeu.
                \noexpand\end{itemize}
            }
    \end{itemize}
    Toutes les sauvegardes sont liées aux profils.

    En dehors des sauvegardes automatiques, le joueur a également la possibilité d'activer le "mode hypothèse" en appuyant sur le bouton checkpoint. 

     \begin{itemize}
        \item \subfonction{Checkpoint}
        {Le ckeckpoint permet de créer une sauvegarde à laquelle le joueur peut revenir quand il le souhaite.}
        {
        Le checkpoint fait partie des sauvegardes mais n'est pas automatique. Celui-ci est activé par le joueur en cliquant sur le bouton dédié. Le joueur rentre ainsi en mode hypothèse qui, comme son nom l'indique, permet au joueur d'essayer des coups sans prendre le risque de ne plus retrouver sa configuration de grille actuelle. Cela lui permet aussi de revenir à la situation actuelle de sa grille sans utiliser plusieurs "retours en arrières" et risquer de se perdre dans son historique de coups. À noter, l'utilisation du checkpoint n'intervient pas sur le chrono du joueur, ainsi rentrer dans le mode hypothèse n'engendre pas de pénalités, mais le retour au checkpoint ne fais pas revenir au chronomètre d'avant checkpoint.
        }
    \end{itemize}
}
{
    La sauvegarde est une fonctionnalité très importante pour un jeu comme le Hashi, elle permet de conserver les avancées, assurant ainsi que les joueurs puissent reprendre leur partie là où ils l'avaient laissée lors des sessions suivantes.
}