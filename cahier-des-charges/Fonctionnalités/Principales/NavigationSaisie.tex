%%%%%%%%%%%%%%%%%%%%%%%%%%%%%%%%%%%%%%%%%
%%%%%%%%%%%%%% PROPOSITION %%%%%%%%%%%%%%
%%%%%%%%%%%%%%%%%%%%%%%%%%%%%%%%%%%%%%%%%
\wPt\hspace{0.02cm}\fonction{Navigation et saisie}
{
    Proposer différents modes de saisie permet au joueur de s'adapter comme bon lui semble.
    \begin{itemize}
        \item \subfonction{Clic sur une île}
            {Permet à l'utilisateur d'afficher le réseau d'une île.}
            {
                Le clic sur une île permet d'afficher le réseau d'une île, c'est à dire toutes les îles auxquelles on peut accéder car relié par un pont à l’île ou à des voisins de lîle.
            }
        \item \subfonction{Clic entre deux îles}
            {Permet à l'utilisateur de créer un pont entre deux îles.}
            {
                Le clic au milieu de deux îles permet de créer un pont entre ces deux îles.
            }
        \item \subfonction{Deux clics entre deux îles}
            {Permet à l'utilisateur de créer deux pont entre deux îles.}
            {
                Le double clic au milieu de deux îles permet de créer deux ponts entre ces deux îles.
            }
        \item \subfonction{Clic entre deux îles disposant déjà d'un ou deux ponts}
            {Permet à l'utilisateur d'ajouter un pont ou de supprimer les ponts}
            {
                Le clic au milieu de deux îles disposant déjà d'un ou deux ponts permet à l'utilisateur d'ajouter ou de supprimer les ponts, conformément aux principe expliqués précédemment:

                Îles disposant d'un pont : le clic ajoute un second pont entre les deux îles \\
                Îles disposant de deux ponts : le clic permet de supprimer les deux ponts et d'atteindre l'état initial entre les deux ponts.
            }
    \end{itemize}
}
{

}
