\fonction{Options}
{
    Plusieurs options sont disponibles:
    \begin{itemize}
    \item \subfonction{Type de saisie}
            {Plusieurs façons d'entrer les valeurs sont disponibles}
            {
                Le joueur aura plusieurs moyens pour entrer les valeurs, avec son clavier, et avec sa souris ainsi qu'un clavier virtuel.
                Il peut selon ses préférences, désactiver un de ces deux modes.
            }
    \end{itemize}

    \begin{itemize}
        \item \subfonction{Thème de l'application}
        {Plusieurs thèmes pour permettre une personnalisation du jeu}
        {
            Pour permettre une personnalisation visuelle de l'application, le joueur aura la possibilité de choisir entre plusieurs thèmes prédéfinis.
        }
    \end{itemize}

    \begin{itemize}
        \item \subfonction{Paramètre de la fenêtre}
        {Choix entre le mode plein écran et le mode fenêtré}
        {
            Le joueur aura la possibilité de choisir si l'application est en plein écran (sans bordures) ou en fenêtré.
        }
    \end{itemize}

    \begin{itemize}
        \item \subfonction{Mode papier}
        {Désactivation de la détection d'infraction des règles}
        {
            Dans ce mode, le joueur n'aura pas la détection d'infraction des règles. Par exemple, le jeu ne va pas indiquer au joueur si il a deux fois le même chiffre dans la même ligne.
        }
    \end{itemize}
    
}
{
    Des options de personnalisation permettent au joueur d'adapter l'application à ses envies:
}