\newpage\wPt\vspace{-2.5cm}
\section{Besoins des joueurs}
Les besoins des joueurs sont définis à partir d’une analyse complète, prenant en compte des aspects tels que l’ergonomie, les attentes spécifiques liées au jeu HashiParmentier et les fonctionnalités complémentaires. Cette démarche garantit une conception adaptée aux attentes et aux préférences des joueurs.

\subsection{Attentes liées au jeu HashiParmentier}
Pour répondre aux attentes liées au jeu Hashi, il est essentiel de distinguer les joueurs en fonction de leur niveau de connaissance.
Ainsi, un joueur débutant, n’ayant jamais joué au Hashi, doit pouvoir apprendre les règles de manière ludique et interactive.\\
Cependant, connaître les règles peut ne pas suffire pour jouer efficacement. C’est pourquoi un système d’aide est intégré, permettant de guider les joueurs novices dans la résolution d’une grille en leur proposant des stratégies adaptées.
Pour mieux s’adapter à la progression des joueurs, il est nécessaire de proposer une classification des grilles selon différents niveaux de difficulté, accompagnée de plusieurs degrés d’assistance.

\subsection{Besoins graphiques et accessibilité}
Les clients souhaitent une interface graphique intuitive, ergonomique, permettant une prise en main rapide de l'application. 

\subsection{Besoins annexes au logiciel}
Pour permettre à plusieurs personnes de jouer sur la même machine, il est nécessaire de proposer une fonctionnalité permettant de créer et de gérer différents profils de joueurs. De plus, pour encourager la comparaison entre joueurs et suivre leur progression dans la résolution des niveaux, l’intégration de classements basés sur un système de scores s’avère particulièrement intéressante.