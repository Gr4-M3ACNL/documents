\newpage\wPt\vspace{-2.5cm}
\section{Besoins des utilisateurs}
Les besoins des utilisateurs se basent sur une analyse complète, couvrant des aspects tels que l'ergonomie, les attentes spécifiques liées au jeu Kakuro, et des fonctionnalités annexes, assurant ainsi une conception répondant précisément aux attentes et préférences des joueurs.

\subsection{Attentes liées au jeu Kakuro}
Concernant les attentes liés au Kakuro, il faut bien distinguer les utilisateurs selon leur niveau de connaissance au jeu. \\
Ainsi, un utilisateur n'ayant jamais joué au Kakuro doit avoir un moyen d'apprendre les règles de façon ludique et immersive.\\
Connaître les règles peut ne pas être suffisant pour pouvoir jouer, un système d'aide est intégré pour guider les utilisateurs novices à compléter une grille en indiquant des stratégies.\\
Afin de s'adapter à la courbe de niveau des joueurs, une catégorisation en niveau de difficulté des grilles est requise ainsi que différents niveaux d'aides.\\

\subsection{Besoins graphique et accessibilité}
Les utilisateurs recherchent une interface graphique intuitive, ergonomique et claire, minimisant le temps nécessaire à la prise en main de l'application. Il est également important d'offrir la possibilité au joueur de personnaliser l'interface pour une expérience plus confortable.

\subsection{Besoins annexes au logiciel}
Pour pouvoir jouer à plusieurs personnes sur la même machine, les utilisateurs ont besoin d'avoir la possibilité de créer et de gérer différents profils utilisateurs. Pour pouvoir se comparer entre eux et jauger leurs avancées dans les différents modes, des classements se basant sur des systèmes de score sont intéressant.