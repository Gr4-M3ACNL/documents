%%%%%%%%%%%%%%%
% PACKAGE %
%%%%%%%%%%%%%%%
% Mise en page
\usepackage{geometry} % Définition des dimension de la page et des marges
\usepackage{fancyhdr} % Gérer la mise en page
\usepackage{lastpage} % Obtenir le numéro de la dernière page
% Titres et TdM
\usepackage{titlesec} % Modifier l'apparence des titres (section, sous-section, etc...)
\usepackage{titletoc} % Modifier l'apparence des titres (section, sous-section, etc...) dans la TdM
% Typographie
\usepackage{babel} % Ob tenir la typographie du francais
\usepackage{libertine} % Change la police pour la police Libertine
\usepackage[T1]{fontenc} % Options de typographies
\usepackage[utf8]{inputenc} % Codage des caractère
\usepackage{enumitem} % Styliser les liste à puces
% Insertion
\usepackage[pdftex]{graphicx} % Ajouter des images
\usepackage{hyperref} % Ajouter des liens
\usepackage{svg} % Ajouter des images vectorielles
\usepackage{longtable} % Faire des tableaux sur plusieurs pages
\usepackage{mdframed} % Faire une minipage sur plusieurs pages
% Couleurs
\usepackage{xcolor} % Ajouter de la couleur
\definecolor{MGCinColor}{RGB}{\MGCrgbColor}
\usepackage{colortbl} % Colorier les tableaux
% LaTeX Programming
\usepackage{ifthen} % Ajout des fonctions conditionnel
\usepackage{pgf} % Ajout de macro dynamique
\usepackage{pgffor} % Ajout de la boucle for
\usepackage{etoolbox} % Ajout des conteneurs de données
\usepackage{import} % Naviger dans les fichiers

% Configuration de l'épaisseur du contour
\usepackage{contour}
\contourlength{1pt}

% Définition d'une commande pour ajouter un point blanc
\newcommand{\wPt}{\textcolor{white}{.}}

%%%%%%%%%%%%%%%
% Informations %
%%%%%%%%%%%%%%%

%%%%%
% Auteurs du document
\newcommand{\auteur}{%
    COGNARD Luka; 
    GAUMONT Maël; 
    LELANDAIS Clément;\\
    MABIRE Aymeric; 
    PESANTEZ Maëlig; 
    PUREN Mewen;\\
    TOUISSI Nassim;
}

%%%%%
% Organisation des auteurs
\newcommand{\organisation}{{M3ACNL}}

% Informations sur l'université et le groupe
\newcommand{\univ}{%
    \textbf{
        {\Huge \textbf{Le Mans Université}}\\
        Licence Informatique\\
        \textit{3ème année}\\
        Génie Logiciel 2 - Gr4\\
        \textbf{Manuel Utilisateur}\\
        \vspace{1.5cm}
        {\color{black}\rule{150pt}{3pt}}\\
        \vspace{1.5cm}
        {\color{white}\contour{black}\organisation}\\
        \wPt
    }
}

%%%%%
% Titre et informations sur le document
\newcommand{\titre}{HashiParmentier} % Titre du document
\newcommand{\executable}{groupe4.jar} % Nom de l'exécutable


\title{\univ\\\textbf{\titre}}
\author{\auteur}

%%%%%%%%%%%%%%%
% Mise en page %
%%%%%%%%%%%%%%%
\/*
%%%%%
% Définire les polices d'écriture
%   % Nomer la nouvelle police
\DeclareFontFamily{T1}{SpringProximaNova}{}
%   % Charger la nouvelle police
\pdfmapline{+SpringProximaNova-Regular <Annexe/Font/Spring-ProximaNova/regular.ttf}
\pdfmapline{+SpringProximaNova-Italic <Annexe/Font/Spring-ProximaNova/regularit.ttf}
\pdfmapline{+SpringProximaNova-Bold <Annexe/Font/Spring-ProximaNova/bold.ttf}
\pdfmapline{+SpringProximaNova-BoldItalic <Annexe/Font/Spring-ProximaNova/boldit.ttf}
%   % Écraser les anciennes polices par défaut
\renewcommand{\rmdefault}{SpringProximaNova}
\renewcommand{\sfdefault}{SpringProximaNova}
\renewcommand{\ttdefault}{SpringProximaNova}
*/

%%%%%
% Définire la taille d'une page
\geometry{
    %a4paper, % Change distance 'gauche'<->'droite' et distance 'haut'<->'bas'
    %textwidth=?cm, % Change distance (8)
    %textheight=?cm, % Change distance (7)
    lmargin=1.5cm, rmargin=1.5cm, % Change distance 'gauche'<->'Corps', Change distance 'Corps'<->'droite'
    tmargin=2cm, bmargin=2cm, % Change distance 'haut'<->'Corps', Change distance 'Corps'<->'bas'
    headheight=1cm, % Change hauteur 'Entête'
    headsep=0.5cm, % Change distance 'Entête'<->'Corps'
    footnotesep=0.5cm, % Change distance 'Corps'<->'Pied de page'
    footskip=0.5cm, % Change distance 'Pied de page'<->'bas'
    marginparwidth=1cm, % Change distance (10)
}
%\usepackage{layout} % Ajouter la commande \layout permettant de voir la taille d'une page

%%%%%
% Définire le contenu des marges
\pagestyle{fancy} % Mettre les options de mise en page à toutes les pages sauf la première
\lhead{\color{MGCinColor}\organisation} % Mettre la liste des auteurs en haut à gauche
\rhead{\color{MGCinColor}\titre} % Mettre le titre du document en haut à droite
\rfoot{\thepage/\pageref{LastPage}} % Mettre le numéro de la page en bas à droite
\renewcommand\headrulewidth{3pt} % Filet de séparation de l'en-tête
\renewcommand{\headrule}{\hbox to\headwidth{\color{MGCinColor}\leaders\hrule height \headrulewidth\hfill}}
\setlength{\headheight}{24pt}

%%%%%
% Définire l'apparence des titres (https://fr.overleaf.com/learn/latex/Sections_and_chapters#titlesec_commands)
%   % \section
\renewcommand{\thesection}{\Roman{section}}
\titleformat{\section}[display]{\normalfont\LARGE\bfseries}{}{0.25cm}
{
    \color{MGCinColor}
    \rule{\textwidth}{1pt}
    \nopagebreak
    \vspace{1ex}
    \nopagebreak
    \centering
    \nopagebreak
    \ifthenelse{\equal{\value{section}}{0}}{}{\thesection{} - }
    \nopagebreak
    \color{black}
}[
    \color{MGCinColor}
    \nopagebreak
    \vspace{-0.6cm}
    \nopagebreak
    \rule{\textwidth}{1pt}
    \nopagebreak
    \color{black}
]
\titlecontents{section}[0.25cm]{}{\contentslabel{0.75cm}}{\hspace*{0.75cm}}{\titlerule*[0.25pc]{.}\contentspage\hspace*{0.25cm}}
%   % \subsection
\titleformat{\subsection}{\normalfont\LARGE\bfseries\color{MGCinColor!90}}{\thesubsection{} )}{0.25cm}{}
\titlecontents{subsection}[1.25cm]{\color{MGCinColor!80}}{\contentslabel{1.25cm}}{\hspace*{0.75cm}}{\titlerule*[0.5pc]{*}\contentspage\hspace*{1.25cm}}
%   % \subsubsection
\titleformat{\subsubsection}{\normalfont\Large\bfseries\color{MGCinColor!60}}{\thesubsubsection{} )}{0.25cm}{}
\titlecontents{subsubsection}[2.25cm]{\color{MGCinColor!60}}{\contentslabel{0.75cm}}{\hspace*{0.75cm}}{\titlerule*[1pc]{-}\contentspage\hspace*{2.25cm}}
%   % \paragraph
\titleformat{\paragraph}[runin]{\normalfont\large\bfseries}{}{}{\underline}[ :\\]
%   % \subparagraph
\titleformat{\subparagraph}[runin]{\normalfont\large\bfseries}{}{}{}[ : ]

%%%%%%%%%%%%%%%
% COMPTEUR %
%%%%%%%%%%%%%%%

%%%%%%%%%%%%%%%
% COMMANDE %
%%%%%%%%%%%%%%%
%%%%%%%%%%%%%%%
% COMPTEUR %
%%%%%%%%%%%%%%%

\newcounter{fonction} % Nom de la dernière fonctionnalité créer
\setcounter{fonction}{0}
\newcounter{subfonction}[fonction] % Nom de la dernière sous-fonctionnalité créer
\setcounter{subfonction}{0}
\newcounter{nbfonction} % Nombre de fonctionnalité et sous-fonctionnalité créer
\setcounter{nbfonction}{0}
\newcounter{idfonction} % Identifiant de la fonctionnalité actuel
\setcounter{idfonction}{0}
\newcounter{debListeFonction} % Liste des description longues des sous-fonctionnalité de la fonctionnalité actuel
\setcounter{debListeFonction}{0}
%%%%%%%%%%%%%%%
% COMMANDE %
%%%%%%%%%%%%%%%

%%%%%
% Ajout d'une fonctionnalité
\newcommand{\descrFonctionList}{}% Initialise la liste de suavegarde des descr
\newcommand{\lireSubFonctionList}{}

\newcommand{\ajouterDescrSousFonction}[2]{%
	\listcsxadd{descrSubFoncList#1}{#2}
}
\newcommand{\codeFonction}[1]{\underline{\textbf{F\arabic{#1} : }}}
\newcommand{\fonction}[3]{%
	% Incrémenter le nombre de fonctionnalité
	\stepcounter{nbfonction}%
	% Incrémenter le compteur dépendant du rang de la fonctionnalité
	\refstepcounter{fonction}%
	% Associer le label à la nouvelle fonction
	\label{fonction:\arabic{nbfonction}}%
	% Créer la liste des description des sous-fonctions
	\setcounter{idfonction}{\arabic{nbfonction}}
	% Afficher la référence de la fonctionnalité et un résumer succint
    \hyperlink{F\arabic{nbfonction}}{\codeFonction{fonction}}\textbf{#1} &%
    %\underline{\textbf{F\thefonction{}}} : #1 &%
	\begin{minipage}{12.5cm}%
        \wPt\\
		#2%
        \wPt\\
	\end{minipage}\\%
	\hline%
	\listxadd{\descrFonctionList}{\noexpand\textbf{{\noexpand\large#1}}\\ #3}%
	\ifthenelse{ \equal{\value{nbfonction}}{\value{idfonction}} }%
		{}{\listxadd{\lireSubFonctionList}{\arabic{idfonction}}}%
}

% Définire l'en-tête du tableau de définition des fonctionnalité.
\newcommand{\enTeteFonctionPrincipal}{
	\rowcolor{MGCinColor}
	\multicolumn{1}{|c|}{ \textbf{{\color{black} Références}} }&
	\multicolumn{1}{|c|}{ \textbf{{\color{black} Fonctionnalités}} }\\
	\hline
}

\newcommand{\obtenirFonctionPrincipal}[1]{%
	% Afficher les fonctionnalité dans un tableau
    \vspace{-0.75cm}
	\begin{center} \begin{longtable}{|l|c|}
        % Première en-tête
        \multicolumn{2}{r}{}\label{tab:fonc} \\
        \hline
		\enTeteFonctionPrincipal%
        \endfirsthead
        % en-tête normal
        \multicolumn{2}{c}%
        {{\bfseries Suite du tableau}} \\
        \hline%
		\enTeteFonctionPrincipal%
        \endhead
        % pied de tableau normal
        \hline \multicolumn{2}{|r|}{{Suite sur la page suivante}} \\ \hline
        \endfoot
        % Dernier pied de tableau normal
		\enTeteFonctionPrincipal%
        \endlastfoot
        % Contenu du tableau
        \wPt\hspace{0.02cm}
		\input{#1}% Lire tous les fichiers de fonctionnalité.
        \vspace{-0.45cm}
	\end{longtable} \end{center}
}
\newcommand{\decrireFonctionPrincipal}{%
	% Lister toutes les fonctionnalité obtenur par \obtenirFonctionPrincipal et les associé à leurs escriptions longues.
    \setcounter{fonction}{0}%
    \setcounter{subfonction}{0}%
	\begin{itemize}%
		\renewcommand*{\do}[1]
		{%
			\stepcounter{debListeFonction}%
			\stepcounter{fonction}%
            \item \hypertarget{F\arabic{debListeFonction}}{\hyperref[fonction:\arabic{debListeFonction}]{\codeFonction{fonction}}}
			##1%
			\xifinlist{\arabic{debListeFonction}}{\lireSubFonctionList}
			{
				\setcounter{idfonction}{\arabic{debListeFonction}}
				\begin{itemize}
					\listerDescrSubFonction
				\end{itemize}
			}{}\wPt \\ \wPt
		}
		\dolistloop{\descrFonctionList}
	\end{itemize}
}

%%%%%
% Ajout d'une sous-fonctionnalité
\newcommand{\descrSubFonctionList}{}%
\newcommand{\codeSubFonction}[2]{\underline{F\arabic{#1}.\arabic{#2} : }}
\newcommand{\subfonction}[3]{%
	% Incrémenter le nombre de fonctionnalité
	\stepcounter{nbfonction}%
	% Incrémenter le compteur dépendant du rang de la fonctionnalité
	\refstepcounter{subfonction}%
	% Associer le label à la nouvelle fonction
	\label{fonction:\arabic{nbfonction}}%
	% Afficher la référence de la fonctionnalité et un résumer succint
    \codeSubFonction{fonction}{subfonction}\textbf{#1} : #2%
	\ajouterDescrSousFonction{\arabic{idfonction}}{\noexpand\textbf{#1}\\ #3}%
}

\newcommand{\listerDescrSubFonction}{
	% Lister toutes les sous-fonctionnalité de cette fonctionnalité et les associer à leur description détailler.
	\renewcommand*{\do}[1]{%
	    \stepcounter{debListeFonction}%
		\stepcounter{subfonction}%
		\item \hyperref[fonction:\arabic{idfonction}]{\codeSubFonction{fonction}{subfonction}}
		##1%
	}%
	\dolistcsloop{descrSubFoncList\arabic{idfonction}}
}

\newcommand{\bouton}[1]{\fcolorbox{black}{gray!25}{#1}}
\/*	Ajouter une nouvelle description dans un environnement `\begin{description}...\end{description}` et le transformer en lien vers l'explication d'une autre maquette
    <bouton> : Le texte du bouton à expliquer
    <label> : Le label (unique) de la maquette à lier
*/
\newcommand{\itemBouton}[2]{ \item[\hyperlink{maquette_#2}{#1} :] }

%%%%%%%%%%%%%%%
% ENVIRONMENT %
%%%%%%%%%%%%%%%

\/*	Ajouter la maquette d'une fenêtre
	#1 : Option pour l'ajout de l'image
	#2 : Nom de l'image (doit se trouver dans le dossier 'Annexe/Maquette' depuis la racine du projet LaTeX)
	#3 : Le nom de l'image
	#4 : L'identifiant de la maquette
	Corps : Description de la fenêtre
    ===============================================================
    Pour le positionnement des images :
    > $https://www.overleaf.com/learn/latex/Positioning_images_and_tables$
*/
\newenvironment{maquette}[4][width=5cm]{%
    {\color{white}\rule{5cm}{0.5pt}}%
    \subsection{#3}\nobreak%
    \begin{figure}[ht]%
		\centering%
  	    \fbox{\includegraphics[#1]{../Annexe/Maquette/#2}}%
		%\caption{ #3 }%
		\label{maquette:#4}%
	\end{figure}%
    \nobreak\wPt\nobreak\\%
    \nobreak\hypertarget{maquette_#4}{\wPt}\nobreak\\%
    \nobreak\begin{minipage}[h]{15cm}%
}{%
	\end{minipage}%
}
%%%%%%%%%%%%%%%
% FIN %
%%%%%%%%%%%%%%%
