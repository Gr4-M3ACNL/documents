%%%%%%%%%%%%%%%%% FONCTIONS UTILES %%%%%%%%%%%%%%%%%%%%%

% \begin{maquette}[<optImage>]{<img>}{<titre>}{<label>}
%   <corp>
%\end{maquette}
\/*	Ajouter la maquette d'une fenêtre
	<optImage> : Option pour l'ajout de l'image (default=[width=5cm])
	<img> : Nom de l'image (doit se trouver dans le dossier 'Annexe/Maquette' depuis la racine du projet LaTeX)
	<titre> : Le nom de l'image (Affichera "Figure <numFig> - <titre>" juste en dessous de l'image)
	<label> : L'identifiant de la maquette (doit être unique (parmi les maquettes))


	Corps : Description de la fenêtre
    ===============================================================
    Pour le positionnement des images :
    > $https://www.overleaf.com/learn/latex/Positioning_images_and_tables$
*/

%\itemBouton{<bouton>}{<label>}
\/*	Ajouter une nouvelle description dans un environnement `\begin{description}...\end{description}` et le transformer en lien vers l'explication d'une autre maquette
    <bouton> : Le texte du bouton à expliquer
    <label> : Le label (unique) de la maquette à lier
*/

%%%%%%%%%%%%%%%%%%%%% DOCUMENT %%%%%%%%%%%%%%%%%%%%%%%%%
\vspace{-1cm}
\section{Maquettes}
\vspace{-1cm}

\begin{maquette}[width=15cm]{logicielTotal.png}{Le logiciel au grand complet}{logicielle}
    L'application Kakuro est conçue avec plusieurs interfaces, offrant à l'utilisateur une expérience complète. Elle propose un tutoriel complet ainsi que des explications guidées pour les nouveaux joueurs. De la création de profils à une expérience de jeu étendue, l'application vise à assurer une immersion optimale.
\end{maquette}

\begin{maquette}{ChoixProfil.png}{Profils}{Profils}
    Lorsque l'application est lancée pour la première fois, il est nécessaire de créer un profil. Depuis les menus du jeu, l'utilisateur peut accéder à une fenêtre dédiée pour gérer tous les profils disponibles. Pour supprimer un profil, il lui suffit de cliquer sur le bouton " - " situé en haut à droite du logo utilisateur. Pour en créer un nouveau, il doit appuyer sur le bouton " + ".
\end{maquette}
\newpage
\begin{maquette}{Menu.png}{Menu d'accueil}{Menu}
        La fenêtre principale dispose de cinq boutons.
        \begin{description}
            \itemBouton{Jouer}{Mode}%text affiché="jouer" | label de la fenêtre ciblé="Mode" 
                Permet d'accéder au menu de choix du mode de jeu.
            \item[Tutoriel :]
                Cette section redirige le joueur vers le tutoriel du jeu.
            \itemBouton{Options}{Options}
                L'engrenage situé en haut à gauche de la fenêtre donne l'accès au menu des options avec la possibilité de changer les paramètres de l'application.
            \itemBouton{Comment jouer}{Commentjouer}
                Le joueur accède à une fenêtre ayant plusieurs explications de l'application et du jeu.
            \itemBouton{Profils}{Profils}
                Le joueur accède à la gestion des profils.
            \item[Quitter :]
                Le bouton pour quitter l'application.
        \end{description}
\end{maquette}

\begin{maquette}{ModeJeu.png}{Mode de jeu}{Mode}
    Sur cette fenêtre, il y a quatre boutons permettant à l'utilisateur de sélectionner un type de jeu.
    \begin{description}
        \itemBouton{Campagne}{Campagne}
            Continuer la campagne du profil actuel.
        \itemBouton{Partie rapide}{Partie}
            Lancer une partie classique.
        \itemBouton{Pixel art}{PixelArt}
            Démarrer une partie de Pixel Art.
        \item[Mode SpeedRun :]
            Démarrer une nouvelle partie.
    \end{description}
\end{maquette}

\begin{maquette}{Explication.png}{Comment jouer}{Commentjouer}
	Liste de nombreuses explications, telle que :
    \begin{itemize}
        \item Les règles de jeu du Kakuro.
        \item Les techniques de résolutions du Kakuro.
        \item L'explication de chaque mode de jeu disponible avec leurs règles.
        \item Les différentes fonctionnalités de l'application.
   \end{itemize}
\end{maquette}

\begin{maquette}{Option.png}{Options}{Options}
	L'option offre la possibilité de changer ses paramètres de jeu :
    \begin{description}
        \item[Type de saisie :] Soit par saisit au clavier lié à l'appareil, ou bien par clique avec un clavier virtuel proposé par l'application.
        \item[Paramètre fenêtre :] Ce menu déroulant permet de choisir une taille de fenêtre par rapport à une liste donnée.
    \end{description}
\end{maquette}

\pagebreak
\begin{maquette}{Campagne.png}{Campagne}{Campagne}
    La fenêtre de la campagne présentera plusieurs niveaux sous forme d'un parcours déroulant.\\
    Le code couleur des niveaux se définit ainsi : 
    \begin{itemize}
        \item Couleur verte : Niveau complété. 
        \item Couleur bleue : Niveau déverrouillé. 
        \item Couleur noire : Niveau verrouillé. 
    \end{itemize}
\end{maquette}

\begin{maquette}{Grille.png}{Interface de jeu}{Grille}
	Le niveau affiche la grille de jeu, elle propose également un bouton d'aide, de vérification et un bouton qui permet de quitter la partie.
    (Voir \hyperref[fonc:interface]{la fonctionnalité interface du jeu} pour plus de détails).
\end{maquette}

\pagebreak
\begin{maquette}{PartieRapide.png}{Partie Classique}{Partie}
	La Partie Classique est une partie personnalisée qui permet à l'utilisateur de choisir sa difficulté et une dimension de grille. L'utilisateur peut également réinitialiser les paramètres par défaut.
\end{maquette}

\begin{maquette}{PixelArt.png}{Pixel Art}{PixelArt}
	La fenêtre PixelArt permet de choisir une dimension qui donnera une grille de pixel art. 
\end{maquette}

\pagebreak
\begin{maquette}{Sauvegarde.png}{Sauvegarde}{Sauvegarde}
    Le menu des sauvegardes affiche l'ensemble des sauvegardes des dernières parties de l'utilisateur, il possède à ce titre plusieurs informations concernant chaque partie : 
    \begin{itemize}
        \item Le mode de jeu.
        \item La date de la dernière partie.
        \item Le pourcentage de complétion.
    \end{itemize}
    L'utilisateur a également la possibilité de reprendre ou de supprimer sa sauvegarde.
\end{maquette}

\begin{maquette}{InfoBulle.png}{Info Bulle}{InfoBulle}
    Voici un exemple d'une fenêtre d'aide dans le jeu. Le joueur a la possibilité de cliquer sur "Indiquer la position" pour obtenir l'aide complémentaire.
\end{maquette}