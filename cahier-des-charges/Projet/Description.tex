\subsection{Description du jeu}
Le puzzle Hashi se présente comme une grille carrée d'îles, dans notre cas représentées par des hachis parmentier, contenant chacune un numéro. 
L'objectif du Hashi est de connecter les îles avec des ponts de manière à respecter quatre règles :
\begin{itemize}
\item \textbf{Solution} : Il doit y avoir un chemin continu parcourant toutes les îles.
\item \textbf{Nombre de ponts} : Il ne peut y avoir plus de deux ponts entre deux îles et le numéro d'une île, correspondant au nombre de ponts qui la relient avec d'autres îles, doit être respecté.
\item \textbf{Orientation des ponts} : Les ponts doivent être horizontaux ou verticaux.
\item \textbf{Croisements} :  Les ponts ne peuvent pas se croiser.
\end{itemize}
Chaque puzzle possède une et une seule solution. Pour relier deux îles, il faut appuyer entre deux îles. Pour supprimer un pont, il suffit de cliquer deux fois entre deux îles, puisqu'un clic correspond à un pont, deux clics à deux ponts (maximum) et le troisième remet à zéro le nombre de ponts.
Il est possible de mettre en surbrillance les îles connectées d'une île en cliquant sur cette île avec la souris. Il est également possible de voir sur quelle(s) île(s) vous pouvez vous connecter en survolant l’île concernée. 
