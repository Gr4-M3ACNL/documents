\subsection{Objectifs du projet}
Le projet HashiParmentier a pour objectif de combiner une collaboration efficace au sein de l’équipe et la création d’un logiciel de qualité, répondant aux attentes des clients. Sur le plan humain, l’objectif prioritaire est de garantir une répartition claire et équilibrée des tâches entre les membres de l’équipe, tout en cultivant une culture de soutien mutuel face aux éventuelles difficultés. La communication régulière sera un levier essentiel pour assurer une coordination optimale et un suivi précis de l’avancement du projet. \\

Au-delà de la simple création d’un jeu, HashiParmentier vise à concevoir une véritable solution réfléchie et éducative. En appliquant rigoureusement les principes du génie logiciel, de la compréhension approfondie des besoins du client à la rédaction d’un cahier des charges détaillé, l’équipe structurera le projet pour maximiser ses chances de succès. 

Sur le plan technique, l’objectif principal est de développer un logiciel qui rend le jeu du Hashi accessible, enrichissant et agréable à tous les joueurs. Plus qu’un simple outil de divertissement, le logiciel doit intégrer un système d’aide interactif et intuitif, pensé pour accompagner les utilisateurs dans leur progression. Que ce soit pour les novices découvrant le jeu ou pour les joueurs expérimentés souhaitant perfectionner leurs techniques, le logiciel offrira une expérience personnalisée et engageante, mettant l’accent sur l’apprentissage et la découverte des techniques de résolution. 
