\subsection{Critères d’acceptabilité}
Les critères d’acceptabilité définissent les normes que le projet \textbf{HashiParmentier} doit respecter pour être jugé réussi :

\begin{enumerate}[label=\textbf{\arabic*.}, leftmargin=1.5cm]

    \item \textbf{Fonctionnalités du jeu} \\
    Le logiciel doit offrir une expérience complète permettant aux joueurs de résoudre des grilles de Hashi avec une difficulté et un système d'aide adapté. 
    \item \textbf{Interface utilisateur (Interface Joueur)} \\
    L’interface doit être conçue de manière à garantir une navigation fluide et agréable. Elle doit être intuitive, bien organisée, et visuellement plaisante pour s’adapter à un large éventail de joueurs. Les interactions doivent être simples et explicites, réduisant au minimum les risques de confusion.

    \item \textbf{Règles du jeu} \\
    Le jeu doit reproduire fidèlement les règles traditionnelles du Hashi. Les mécanismes de jeu doivent être précis et cohérents, afin de garantir une expérience conforme à ce que les clients comme les joueurs attendent.

    \item \textbf{Documentation} \\
    Une documentation claire et détaillée doit accompagner le logiciel. Elle doit inclure un guide d’utilisation expliquant les fonctionnalités principales et un ensemble d’instructions pour résoudre les éventuels problèmes ou questions rencontrés par les joueurs.

\end{enumerate}

Ces critères permettront de garantir un logiciel fonctionnel, pratique et apprécié par les joueurs et correspondant aux attentes des clients.