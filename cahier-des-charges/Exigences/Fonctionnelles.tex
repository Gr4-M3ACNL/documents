\subsection{Exigences fonctionnelles}
Les exigences fonctionnelles du projet se répartissent en plusieurs catégories, groupant divers aspects pour assurer la qualité et la diversité de l'expérience utilisateur.

\paragraph{Exigences sur l'interface graphique}
\begin{itemize}
    \item Permettre à l'utilisateur de naviguer à la souris comme au clavier.
    \item Rendre l'application claire, intuitive et ergonomique en suivant les normes d'UI-UX design.
    \item Faire des pages accessibles, facilitant un accès rapide aux grilles du jeu.
    \item Assurer que les utilisateurs aient toujours la possibilité de retourner à la fenêtre précédente, au menu d'accueil, à la fenêtre d'options et aux règles du jeu.
    \item Rendre l'application personnalisable en offrant plusieurs thèmes graphiques, permettant aux utilisateurs de choisir celui qui leur convient le mieux.
\end{itemize}

\paragraph{Exigences liées au jeu Kakuro}
\begin{itemize}
    \item Créer des variantes au jeu en proposant plusieurs modes de jeu.
    \item Adapter le niveau du jeu en fournissant plusieurs moyens d'apprendre à jouer, dont des explications des règles, des techniques de résolution, et un tutoriel.
    \item Offrir des aides couvrant différentes notions telles que l'aide pour indiquer une technique de jeu applicable, la vérification des valeurs, et la signalisation des valeurs impossibles.
    \item Laisser à l'utilisateur la possibilité de désactiver toutes ces formes d'aide.
    \item Classer chaque grille comme 'facile', 'normal' ou 'difficile', permettant au joueur de choisir la difficulté au début de chaque partie.
\end{itemize}

\paragraph{Exigences annexes au logiciel}
\begin{itemize}
    \item Mettre en place une fonctionnalité de sauvegarde permettant de conserver les parties d'un joueur.
    \item Autoriser l'existence de plusieurs profils, afin de permettre à plusieurs utilisateurs d'avoir leur propre aventure sur la même machine.
    \item Comparer les joueurs entre eux en mettant en place des classements et des tableaux des records personnels.
\end{itemize}
