%%%%%%%%%%%%%%%
% PACKAGE %
%%%%%%%%%%%%%%%
% Mise en page
\usepackage{geometry} % Définition des dimensions de la page et des marges
\usepackage{fancyhdr} % Gérer la mise en page des en-têtes et pieds de page
\usepackage{lastpage} % Obtenir le numéro de la dernière page
% Titres et TdM
\usepackage{titlesec} % Modifier l'apparence des titres (section, sous-section, etc...)
\usepackage{titletoc} % Modifier l'apparence des titres dans la table des matières
% Typographie
\usepackage{babel} % Obtenir la typographie en français
\usepackage{libertine} % Change la police pour la police Libertine
\usepackage[T1]{fontenc} % Options de typographies
\usepackage[utf8]{inputenc} % Codage des caractères
\usepackage{enumitem} % Styliser les listes à puces
% Insertion
\usepackage[pdftex]{graphicx} % Ajouter des images
\usepackage{hyperref} % Ajouter des liens
\usepackage{svg} % Ajouter des images vectorielles
\usepackage{longtable} % Faire des tableaux sur plusieurs pages
\usepackage{mdframed} % Faire une minipage sur plusieurs pages
% Couleurs
\usepackage{xcolor} % Ajouter de la couleur
\definecolor{MGCinColor}{RGB}{0.0, 0.0, 0.0}
\usepackage{colortbl} % Colorier les tableaux
% LaTeX Programming
\usepackage{ifthen} % Ajout des fonctions conditionnelles
\usepackage{pgf} % Ajout de macros dynamiques
\usepackage{pgffor} % Ajout de la boucle for
\usepackage{etoolbox} % Ajout des conteneurs de données
\usepackage{import} % Naviguer dans les fichiers

% Configuration de l'épaisseur du contour
\usepackage{contour}
\contourlength{1pt}

% Définition d'une commande pour ajouter un point blanc
\newcommand{\wPt}{\textcolor{white}{.}}

%%%%%%%%%%%%%%%
% Informations %
%%%%%%%%%%%%%%%

%%%%%
% Auteurs du document
\newcommand{\auteur}{%
    COGNARD Luka; 
    GAUMONT Maël; 
    LELANDAIS Clément;\\
    MABIRE Aymeric; 
    PESANTEZ Maëlig; 
    PUREN Mewen;\\
    TOUISSI Nassim;
}

%%%%%
% Organisation des auteurs
\newcommand{\organisation}{{M3ACNL}}

% Informations sur l'université et le groupe
\newcommand{\univ}{%
    \textbf{
        {\Huge \textbf{Le Mans Université}}\\
        Licence Informatique\\
        \textit{3ème année}\\
        Génie Logiciel 2 - Gr4\\
        \textbf{Manuel Utilisateur}\\
        \vspace{1.5cm}
        {\color{black}\rule{150pt}{3pt}}\\
        \vspace{1.5cm}
        {\color{white}\contour{black}\organisation}\\
        \wPt
    }
}

%%%%%
% Titre et informations sur le document
\newcommand{\titre}{HashiParmentier} % Titre du document
\newcommand{\executable}{groupe4.jar} % Nom de l'exécutable


\title{\univ\\\textbf{\titre}}
\author{\auteur}

%%%%%%%%%%%%%%%
% Mise en page %
%%%%%%%%%%%%%%%
\geometry{
    %a4paper, % Change distance 'gauche'<->'droite' et distance 'haut'<->'bas'
    %textwidth=?cm, % Change distance (8)
    %textheight=?cm, % Change distance (7)
    lmargin=1.5cm, rmargin=1.5cm, % Change distance 'gauche'<->'Corps', Change distance 'Corps'<->'droite'
    tmargin=2cm, bmargin=2cm, % Change distance 'haut'<->'Corps', Change distance 'Corps'<->'bas'
    headheight=1cm, % Change hauteur 'Entête'
    headsep=0.5cm, % Change distance 'Entête'<->'Corps'
    footnotesep=0.5cm, % Change distance 'Corps'<->'Pied de page'
    footskip=0.5cm, % Change distance 'Pied de page'<->'bas'
    marginparwidth=1cm, % Change distance (10)
}
%\usepackage{layout} % Ajouter la commande \layout permettant de voir la taille d'une page

%%%%%
% Définire le contenu des marges
\pagestyle{fancy} % Mettre les options de mise en page à toutes les pages sauf la première
\lhead{\color{MGCinColor}\organisation} % Mettre la liste des auteurs en haut à gauche
\rhead{\color{MGCinColor}\titre} % Mettre le titre du document en haut à droite
\rfoot{\thepage/\pageref{LastPage}} % Mettre le numéro de la page en bas à droite
\renewcommand\headrulewidth{3pt} % Filet de séparation de l'en-tête
\renewcommand{\headrule}{\hbox to\headwidth{\color{MGCinColor}\leaders\hrule height \headrulewidth\hfill}}
\setlength{\headheight}{24pt}

%%%%%
% Définire l'apparence des titres (https://fr.overleaf.com/learn/latex/Sections_and_chapters#titlesec_commands)
%   % \section
\renewcommand{\thesection}{\Roman{section}}
\titleformat{\section}[display]{\normalfont\LARGE\bfseries}{}{0.25cm}
{
    \color{MGCinColor}
    \rule{\textwidth}{1pt}
    \nopagebreak
    \vspace{1ex}
    \nopagebreak
    \centering
    \nopagebreak
    \ifthenelse{\equal{\value{section}}{0}}{}{\thesection{} - }
    \nopagebreak
    \color{black}
}[
    \color{MGCinColor}
    \nopagebreak
    \vspace{-0.6cm}
    \nopagebreak
    \rule{\textwidth}{1pt}
    \nopagebreak
    \color{black}
]
\titlecontents{section}[0.25cm]{}{\contentslabel{0.75cm}}{\hspace*{0.75cm}}{\titlerule*[0.25pc]{.}\contentspage\hspace*{0.25cm}}
%   % \subsection
\titleformat{\subsection}{\normalfont\LARGE\bfseries\color{MGCinColor!90}}{\thesubsection{} )}{0.25cm}{}
\titlecontents{subsection}[1.25cm]{\color{MGCinColor!80}}{\contentslabel{1.25cm}}{\hspace*{0.75cm}}{\titlerule*[0.5pc]{*}\contentspage\hspace*{1.25cm}}
%   % \subsubsection
\titleformat{\subsubsection}{\normalfont\Large\bfseries\color{MGCinColor!60}}{\thesubsubsection{} )}{0.25cm}{}
\titlecontents{subsubsection}[2.25cm]{\color{MGCinColor!60}}{\contentslabel{0.75cm}}{\hspace*{0.75cm}}{\titlerule*[1pc]{-}\contentspage\hspace*{2.25cm}}
%   % \paragraph
\titleformat{\paragraph}[runin]{\normalfont\large\bfseries}{}{}{\underline}[ :\\]
%   % \subparagraph
\titleformat{\subparagraph}[runin]{\normalfont\large\bfseries}{}{}{}[ : ]

%%%%%%%%%%%%%%%
% COMPTEUR %
%%%%%%%%%%%%%%%

%%%%%%%%%%%%%%%
% COMMANDE %
%%%%%%%%%%%%%%%
\newcommand{\bouton}[1]{\fcolorbox{black}{gray!25}{#1}}
\/*	Ajouter une nouvelle description dans un environnement `\begin{description}...\end{description}` et le transformer en lien vers l'explication d'une autre maquette
    <bouton> : Le texte du bouton à expliquer
    <label> : Le label (unique) de la maquette à lier
*/
\newcommand{\itemBouton}[2]{ \item[\hyperlink{maquette_#2}{#1} :] }

%%%%%%%%%%%%%%%
% ENVIRONMENT %
%%%%%%%%%%%%%%%

\/*	Ajouter la maquette d'une fenêtre
	#1 : Option pour l'ajout de l'image
	#2 : Nom de l'image (doit se trouver dans le dossier 'Annexe/Maquette' depuis la racine du projet LaTeX)
	#3 : Le nom de l'image
	#4 : L'identifiant de la maquette
	Corps : Description de la fenêtre
    ===============================================================
    Pour le positionnement des images :
    > $https://www.overleaf.com/learn/latex/Positioning_images_and_tables$
*/
\newenvironment{maquette}[4][width=5cm]{%
    {\color{white}\rule{5cm}{0.5pt}}%
    \subsection{#3}\nobreak%
    \begin{figure}[ht]%
		\centering%
  	    \fbox{\includegraphics[#1]{../Annexe/Maquette/#2}}%
		%\caption{ #3 }%
		\label{maquette:#4}%
	\end{figure}%
    \nobreak\wPt\nobreak\\%
    \nobreak\hypertarget{maquette_#4}{\wPt}\nobreak\\%
    \nobreak\begin{minipage}[h]{15cm}%
}{%
	\end{minipage}%
}
%%%%%%%%%%%%%%%
% FIN %
%%%%%%%%%%%%%%%